% Options for packages loaded elsewhere
\PassOptionsToPackage{unicode}{hyperref}
\PassOptionsToPackage{hyphens}{url}
%
\documentclass[
  french,
]{article}
\usepackage{amsmath,amssymb}
\usepackage{lmodern}
\usepackage{ifxetex,ifluatex}
\ifnum 0\ifxetex 1\fi\ifluatex 1\fi=0 % if pdftex
  \usepackage[T1]{fontenc}
  \usepackage[utf8]{inputenc}
  \usepackage{textcomp} % provide euro and other symbols
\else % if luatex or xetex
  \usepackage{unicode-math}
  \defaultfontfeatures{Scale=MatchLowercase}
  \defaultfontfeatures[\rmfamily]{Ligatures=TeX,Scale=1}
\fi
% Use upquote if available, for straight quotes in verbatim environments
\IfFileExists{upquote.sty}{\usepackage{upquote}}{}
\IfFileExists{microtype.sty}{% use microtype if available
  \usepackage[]{microtype}
  \UseMicrotypeSet[protrusion]{basicmath} % disable protrusion for tt fonts
}{}
\makeatletter
\@ifundefined{KOMAClassName}{% if non-KOMA class
  \IfFileExists{parskip.sty}{%
    \usepackage{parskip}
  }{% else
    \setlength{\parindent}{0pt}
    \setlength{\parskip}{6pt plus 2pt minus 1pt}}
}{% if KOMA class
  \KOMAoptions{parskip=half}}
\makeatother
\usepackage{xcolor}
\IfFileExists{xurl.sty}{\usepackage{xurl}}{} % add URL line breaks if available
\IfFileExists{bookmark.sty}{\usepackage{bookmark}}{\usepackage{hyperref}}
\hypersetup{
  pdftitle={FINAL},
  pdfauthor={KASHALA ILUNGA Caleb},
  pdflang={fr},
  hidelinks,
  pdfcreator={LaTeX via pandoc}}
\urlstyle{same} % disable monospaced font for URLs
\usepackage[margin=1in]{geometry}
\usepackage{graphicx}
\makeatletter
\def\maxwidth{\ifdim\Gin@nat@width>\linewidth\linewidth\else\Gin@nat@width\fi}
\def\maxheight{\ifdim\Gin@nat@height>\textheight\textheight\else\Gin@nat@height\fi}
\makeatother
% Scale images if necessary, so that they will not overflow the page
% margins by default, and it is still possible to overwrite the defaults
% using explicit options in \includegraphics[width, height, ...]{}
\setkeys{Gin}{width=\maxwidth,height=\maxheight,keepaspectratio}
% Set default figure placement to htbp
\makeatletter
\def\fps@figure{htbp}
\makeatother
\setlength{\emergencystretch}{3em} % prevent overfull lines
\providecommand{\tightlist}{%
  \setlength{\itemsep}{0pt}\setlength{\parskip}{0pt}}
\setcounter{secnumdepth}{5}
\usepackage{booktabs}
\usepackage{longtable}
\usepackage{array}
\usepackage{multirow}
\usepackage{wrapfig}
\usepackage{float}
\usepackage{colortbl}
\usepackage{pdflscape}
\usepackage{tabu}
\usepackage{threeparttable}
\usepackage{threeparttablex}
\usepackage[normalem]{ulem}
\usepackage{makecell}
\usepackage{xcolor}
\ifxetex
  % Load polyglossia as late as possible: uses bidi with RTL langages (e.g. Hebrew, Arabic)
  \usepackage{polyglossia}
  \setmainlanguage[]{french}
\else
  \usepackage[main=french]{babel}
% get rid of language-specific shorthands (see #6817):
\let\LanguageShortHands\languageshorthands
\def\languageshorthands#1{}
\fi
\ifluatex
  \usepackage{selnolig}  % disable illegal ligatures
\fi

\title{FINAL}
\author{KASHALA ILUNGA Caleb}
\date{01/10/2019}

\begin{document}
\maketitle

{
\setcounter{tocdepth}{2}
\tableofcontents
}
\newpage

\hypertarget{descriptifs-des-variables}{%
\subsection{Descriptifs des variables}\label{descriptifs-des-variables}}

\begin{itemize}
\tightlist
\item
  \texttt{FID} : Identifiant;
\item
  \texttt{OBJECTID} : Identifiant ;
\item
  \texttt{AmbulanceR} : Prise en charge par une ambulance ;
\item
  \texttt{BikeAge\_Gr} : Tranche d'age du cycliste ;
\item
  \texttt{Bike\_Age} : Age du cycliste ;
\item
  \texttt{Bike\_Alc\_D} : Alcooléémie du cycliste ;
\item
  \texttt{Bike\_Dir} : Direction du cycliste ;
\item
  \texttt{Bike\_Injur} : Bléssure du cycliste ;
\item
  \texttt{Bike\_Pos} : Position du cycliste ;
\item
  \texttt{Bike\_Race} : La race du cycliste ;
\item
  \texttt{Bike\_Sex} : Le sexe du cycliste ;
\item
  \texttt{City} : La Ville;
\item
  \texttt{County} : Comté ;
\item
  \texttt{CrashAlcoh} : Si une des deux personnes était alcoolisé ;
\item
  \texttt{CrashDay} : S'il a déjà eu un accident de travail ;
\item
  \texttt{Crash\_Date} : Date de l'accident ;
\item
  \texttt{Crash\_Grp} : La cause de l'accident ;
\item
  \texttt{Crash\_Hour} : L'heure de l'accident ;
\item
  \texttt{Crash\_Loc} : S'il y avait une intersection ou pas ;
\item
  \texttt{Crash\_Mont} : ;Le mois de l'accident ;
\item
  \texttt{Crash\_Time} : La date de l'accident ;
\item
  \texttt{Crash\_Type} : Type de l'accident ;
\item
  \texttt{Crash\_Ty\_1} : ;
\item
  \texttt{Crash\_Year} : L'année de l'accident ;
\item
  \texttt{Crsh\_Sevri} : La séverité de la blessure ;
\item
  \texttt{Development} : Le cadre de la région dans lequel l'accident
  s'est déroulé ;
\item
  \texttt{DrvrAge\_Gr} : La tranche d'age du conducteur qui a fait
  l'accident ;
\item
  \texttt{Drvr\_Age} : L'age du conducteur ;
\item
  \texttt{Drvr\_Alc\_D} : Si le conducteur était alcoolisé ou pas ;
\item
  \texttt{Drvr\_EstSp} : Vitesse du conducteur responsable de l'accident
  ;
\item
  \texttt{Drvr\_Injur} : L'etat du conducteur responsable de l'accident
  ;
\item
  \texttt{Drvr\_Race} : La race du conducteur responsable de l'accident
  ;
\item
  \texttt{Drvr\_Sex} : Le sexe du conducteur responsable de l'accident ;
\item
  \texttt{Drvr\_VehTy} : Le type de véhicule ;
\item
  \texttt{ExcsSpdInd} : Depassement de la vitesse limite autorisée;
\item
  \texttt{Hit\_Run} : Si le conducteur s'est enfuit sans vérifier que le
  pilote va bien ;
\item
  \texttt{Light\_Cond} : La luminosité ;
\item
  \texttt{Locality} : Localité ;
\item
  \texttt{Num\_Lanes} : Numéro de la voie ;
\item
  \texttt{Num\_Units} : Numéro de l'unité ;
\item
  \texttt{Rd\_Charact} : ;
\item
  \texttt{Rd\_Class} : Le type de route ;
\item
  \texttt{Rd\_Conditi} : Condition de la route ;
\item
  \texttt{Rd\_Config} : Le genre de la route ;
\item
  \texttt{Rd\_Defects} : L'Etat de la route ;
\item
  \texttt{Rd\_Feature} : Spécificité de la route ;
\item
  \texttt{Rd\_Surface} : La surface de la route ;
\item
  \texttt{Region} : La Region ;
\item
  \texttt{Rural\_Urba} : Rurale ou urbaine ;
\item
  \texttt{Speed\_Limi} : Limite de vitesse ;
\item
  \texttt{Traff\_Cntr} : Controle du trafique ;
\item
  \texttt{Weather} : Condition météo ;
\item
  \texttt{Workzone\_I} : Zone de travail ;
\item
  \texttt{Location} : Localisation ;
\end{itemize}

Pour notre étude analytique nous allons travailler sur une base de
données qui nous donne un ensemble d'informations sur les accidents de
vélo qui se sont produit aux Etats-Unis. Elle nous donne des
informations sur des variables innhérentes aux conducteurs impliqués
dans ces accidents mais aussi sur les divers conditions et situations
physiques et environnementales qui l'entoure. Cette base de données nous
renseigne sur 5716 observations contenant 54 variables.

\#Plan

Pour mener à bien notre analyse nous allons procédés en plusieurs etapes
: * \texttt{Etudes\ sur\ les\ cyclistes} : consacré à l'etudes des
varaiables qui concerne le cycliste ; *
\texttt{Etude\ sur\ les\ conducteurs} : consacré à l'etudes des
variables qui concerne les conducteurs ; *
\texttt{Etudes\ sur\ les\ données\ temporelles} : consacré à l'etudes
des variables qui concerne le cycliste ; *
\texttt{Etude\ sur\ les\ condtions\ de\ l\textquotesingle{}accident} :
qui correpond aux données sur les conditions météo, l'etat de la route
etc.. ;

\begin{itemize}
\tightlist
\item
  \texttt{Etude\ sur\ les\ autres\ variables} : qui correspond aux
  données non classeés mais qui semblent pertinentes ;
\end{itemize}

\#Données manquantes

\includegraphics{Analyse-statistique_files/figure-latex/unnamed-chunk-3-1.pdf}

\begin{verbatim}
## 
##  Variables sorted by number of missings: 
##    Variable        Count
##    Drvr_Age 0.1917424773
##  Crash_Ty_1 0.1784464661
##    CrashDay 0.1445066480
##  DrvrAge_Gr 0.1327851645
##    Drvr_Sex 0.1326102169
##  Drvr_VehTy 0.0883484955
##  Drvr_EstSp 0.0831000700
##   Num_Lanes 0.0558082575
##  Speed_Limi 0.0502099370
##    Bike_Dir 0.0489853044
##    Bike_Pos 0.0442617215
##  Crash_Mont 0.0411126662
##    Bike_Age 0.0227431770
##  BikeAge_Gr 0.0195941218
##    Bike_Sex 0.0167949615
##   Rd_Config 0.0101469559
##    Rd_Class 0.0052484255
##  Rd_Charact 0.0050734780
##  Rd_Defects 0.0045486354
##  Rd_Surface 0.0033240028
##  Rd_Conditi 0.0029741078
##  Light_Cond 0.0013995801
##   Crash_Grp 0.0006997901
##   Crash_Loc 0.0006997901
##  Crash_Type 0.0006997901
##         FID 0.0000000000
##    OBJECTID 0.0000000000
##  AmbulanceR 0.0000000000
##  Bike_Alc_D 0.0000000000
##  Bike_Injur 0.0000000000
##   Bike_Race 0.0000000000
##        City 0.0000000000
##      County 0.0000000000
##  CrashAlcoh 0.0000000000
##  Crash_Date 0.0000000000
##  Crash_Hour 0.0000000000
##  Crash_Time 0.0000000000
##  Crash_Year 0.0000000000
##  Crsh_Sevri 0.0000000000
##  Developmen 0.0000000000
##  Drvr_Alc_D 0.0000000000
##  Drvr_Injur 0.0000000000
##   Drvr_Race 0.0000000000
##  ExcsSpdInd 0.0000000000
##     Hit_Run 0.0000000000
##    Locality 0.0000000000
##   Num_Units 0.0000000000
##  Rd_Feature 0.0000000000
##      Region 0.0000000000
##  Rural_Urba 0.0000000000
##  Traff_Cntr 0.0000000000
##     Weather 0.0000000000
##  Workzone_I 0.0000000000
##    Location 0.0000000000
\end{verbatim}

Les variables qui contiennent le plus de données manquantes sont : L'age
du vhauffeur, Crash\_ty\_1, Le jour de du crash , et le sexe du
chauffeur.

\hypertarget{etude-sur-les-cycliste}{%
\section{Etude sur les cycliste}\label{etude-sur-les-cycliste}}

\hypertarget{age-et-sexe-des-cyclistes}{%
\subsection{Age et Sexe des cyclistes}\label{age-et-sexe-des-cyclistes}}

\includegraphics{Analyse-statistique_files/figure-latex/unnamed-chunk-5-1.pdf}

Nous remarquons sur ce graphique que les accidents de vélo concernent
essentiellements les hommes et que la moyennes d'age des accidentées
etait relativement la meme quelque soit le sexe.

\includegraphics{Analyse-statistique_files/figure-latex/unnamed-chunk-6-1.pdf}
~

L'age des accidentés varie fortement, cependant on peut constater sur ce
graphique que le nombre d'accidents est beaucoup plus élevé pour les
individu ayant entre 13 et 21 ans ; Puis s'en suis une forte baise du
nombre d'accidents chez les 22-40 ans puis, après une legère
augmentation chez les 41-53 ans, le nombre d'accident se réduit
considérablement.

~

\hypertarget{alcooluxe9mie}{%
\subsection{Alcoolémie}\label{alcooluxe9mie}}

\includegraphics{Analyse-statistique_files/figure-latex/unnamed-chunk-7-1.pdf}
~

La plupart des accidents ne sont pas dût à l'alccol.

\hypertarget{race-du-cycliste}{%
\subsection{Race du cycliste}\label{race-du-cycliste}}

\begin{table}
\centering
\begin{tabular}{l|r|r}
\hline
  & nombre & Proportion\\
\hline
White & 3111 & 0.5442617\\
\hline
Other & 48 & 0.0083975\\
\hline
native American & 67 & 0.0117215\\
\hline
Hispanic & 297 & 0.0519594\\
\hline
Black & 2006 & 0.3509447\\
\hline
Asian & 56 & 0.0097971\\
\hline
Missing & 131 & 0.0229181\\
\hline
\end{tabular}
\end{table}

\includegraphics{Analyse-statistique_files/figure-latex/unnamed-chunk-10-1.pdf}

\hypertarget{bluxe9ssures-du-cycliste.}{%
\subsection{Bléssures du cycliste.}\label{bluxe9ssures-du-cycliste.}}

\begin{table}
\centering
\begin{tabular}{l|r}
\hline
Bike\_Injur & Freq\\
\hline
B: Evident Injury & 2405\\
\hline
C: Possible Injury & 2199\\
\hline
O: No Injury & 526\\
\hline
A: Disabling Injury & 291\\
\hline
Injury & 172\\
\hline
K: Killed & 123\\
\hline
\end{tabular}
\end{table}

\begin{verbatim}
## Warning: Use of `Injur$Bike_Injur` is discouraged. Use `Bike_Injur` instead.
\end{verbatim}

\begin{verbatim}
## Warning: Use of `Injur$Freq` is discouraged. Use `Freq` instead.
\end{verbatim}

\includegraphics{Analyse-statistique_files/figure-latex/unnamed-chunk-12-1.pdf}

La majeure partie des accidents, bien que n'étant pas mortel, entraine
dans la plupart des cas des blessures graves. Ceci s'explique notament
parce que la plupart des colision se font avec des automiblistes et que
les cyclistes n'ont aucune protections d'aucune sorte.

\hypertarget{relation-entre-type-de-blessure-et-position-du-vuxe9lo}{%
\subsection{Relation entre type de blessure et Position du
vélo}\label{relation-entre-type-de-blessure-et-position-du-vuxe9lo}}

\begin{table}
\centering
\begin{tabular}{l|r}
\hline
Var1 & Freq\\
\hline
Travel Lane & 3746\\
\hline
Sidewalk / Crosswalk / Driveway Crossing & 899\\
\hline
Non-Roadway & 309\\
\hline
Bike Lane / Paved Shoulder & 295\\
\hline
Driveway / Alley & 146\\
\hline
Multi-use Path & 36\\
\hline
Other & 32\\
\hline
\end{tabular}
\end{table}

\begin{verbatim}
## Warning: Use of `bike$Bike_Injur` is discouraged. Use `Bike_Injur` instead.
\end{verbatim}

\begin{verbatim}
## Warning: Use of `bike$Bike_Pos` is discouraged. Use `Bike_Pos` instead.
\end{verbatim}

\includegraphics{Analyse-statistique_files/figure-latex/unnamed-chunk-14-1.pdf}

\hypertarget{etude-sur-les-conducteurs}{%
\section{Etude sur les conducteurs}\label{etude-sur-les-conducteurs}}

\hypertarget{sexe-du-conducteur}{%
\subsection{Sexe du conducteur}\label{sexe-du-conducteur}}

\begin{table}
\centering
\begin{tabular}{l|r}
\hline
Var1 & Freq\\
\hline
Female & 2231\\
\hline
Male & 2727\\
\hline
\end{tabular}
\end{table}

\includegraphics{Analyse-statistique_files/figure-latex/unnamed-chunk-16-1.pdf}

~

Contrairement aux accidentés qui sont essentiellement des hommes, le
sexe des auteurs des accidents est quand à lui repartit plus
quitablement entre les hommes et les femmes. Cependant le doute subsiste
etant donné le nombre de données manquantes.

\hypertarget{age-des-chauffeurs.}{%
\subsection{Age des chauffeurs.}\label{age-des-chauffeurs.}}

\includegraphics{Analyse-statistique_files/figure-latex/unnamed-chunk-17-1.pdf}
~

L'age du chauffeur est la variable ayant le plus de données manquantes.
Cette representation graphique permet néanmoins d'observer la
distribution de l'age des chauffeurs. Trois pics sont a notés, le
premier a 21 ans , le second à 40 et enfin le dernier a 60 ans.

\hypertarget{bluxe9ssures-des-chauffeurs}{%
\subsection{Bléssures des
chauffeurs}\label{bluxe9ssures-des-chauffeurs}}

\begin{table}
\centering
\begin{tabular}{l|r}
\hline
Var1 & Freq\\
\hline
O: No Injury & 0.8341498\\
\hline
Injury & 0.1410077\\
\hline
C: Possible Injury & 0.0143457\\
\hline
B: Evident Injury & 0.0094472\\
\hline
A: Disabling Injury & 0.0006998\\
\hline
K: Killed & 0.0003499\\
\hline
\end{tabular}
\end{table}

\includegraphics{Analyse-statistique_files/figure-latex/unnamed-chunk-19-1.pdf}

On aurait pu supposer d'intinct qu'un accident entre une voiture et un
cycliste ne causeraI que très peu de dommage au conducteur de la
voiture, les données le confirme. Dans la plupart des accidents le
chauffeur s'en est sorti sans blessures. Il n'y a que 2 morts parmi les
5716 accidents recensés.

\hypertarget{relation-entre-la-vitesse-du-chaufeur-sa-couleur-de-peau-et-le-nombre-daccidents}{%
\subsection{Relation entre la vitesse du chaufeur, sa couleur de peau et
le nombre
d'accidents}\label{relation-entre-la-vitesse-du-chaufeur-sa-couleur-de-peau-et-le-nombre-daccidents}}

\begin{table}
\centering
\begin{tabular}{l|r|r|r|r|r|r|r}
\hline
  & /Missing & Asian & Black & Hispanic & Native American & Other & White\\
\hline
>100 mph & 0 & 0 & 0 & 0 & 0 & 0 & 2\\
\hline
0-5 mph & 114 & 14 & 317 & 39 & 5 & 14 & 819\\
\hline
11-15 mph & 42 & 5 & 123 & 15 & 2 & 7 & 213\\
\hline
16-20 mph & 48 & 5 & 123 & 9 & 2 & 5 & 222\\
\hline
21-25 mph & 43 & 4 & 142 & 8 & 4 & 6 & 214\\
\hline
26-30 mph & 22 & 1 & 91 & 15 & 5 & 2 & 150\\
\hline
31-35 mph & 92 & 3 & 201 & 20 & 6 & 6 & 324\\
\hline
36-40 mph & 21 & 3 & 44 & 10 & 2 & 2 & 113\\
\hline
41-45 mph & 74 & 1 & 99 & 7 & 7 & 3 & 309\\
\hline
46-50 mph & 8 & 1 & 23 & 3 & 0 & 2 & 70\\
\hline
51-55 mph & 42 & 1 & 50 & 6 & 20 & 1 & 158\\
\hline
56-60 mph & 0 & 0 & 4 & 5 & 0 & 0 & 13\\
\hline
6-10 mph & 74 & 11 & 175 & 20 & 1 & 6 & 340\\
\hline
61-65 mph & 1 & 1 & 1 & 0 & 0 & 0 & 2\\
\hline
66-70 mph & 0 & 0 & 1 & 0 & 0 & 0 & 1\\
\hline
81-85 mph & 0 & 0 & 0 & 0 & 0 & 0 & 1\\
\hline
\end{tabular}
\end{table}

\includegraphics{Analyse-statistique_files/figure-latex/unnamed-chunk-21-1.pdf}

Les chauffeurs les plus impliqués dans un accidents sont de race
Blanche, les second etant Noir. La plupart des individus impliqués dans
un accident roulent entre 0 et 5 mph (milles par heure). Le fait que les
victimes et les auteurs des accidents soient en majorité blanche peut
s'expliquer simplement par le fait que la population soit en majorité
blanche. Il serait intéréssant de rapporter ces données a la proportion
de la population, mais nous manquons de données pour cela. Notons aussi
que nous avant 1500 accidents dont nous n'avons pas de données sur la
vitesse.

\#\#Alcoolémie

\includegraphics{Analyse-statistique_files/figure-latex/unnamed-chunk-22-1.pdf}

\begin{wraptable}{r}{0pt}
\begin{tabular}{l|r}
\hline
Var1 & Freq\\
\hline
Missing & 632\\
\hline
No & 4990\\
\hline
Yes & 94\\
\hline
\end{tabular}\end{wraptable}

La plupart des chauffeurs n'etaient pas alcoolisés.

\hypertarget{type-de-voitures}{%
\subsection{Type de voitures}\label{type-de-voitures}}

\includegraphics{Analyse-statistique_files/figure-latex/unnamed-chunk-24-1.pdf}

\begin{table}
\centering
\begin{tabular}{l|r}
\hline
Var1 & Freq\\
\hline
Passenger Car & 3032\\
\hline
Sport Utility & 823\\
\hline
Pickup & 745\\
\hline
Van & 294\\
\hline
Light Truck (Mini-Van, Panel) & 85\\
\hline
Single Unit Truck (2-Axle, 6-Tire) & 40\\
\hline
Motorcycle & 36\\
\hline
Police & 35\\
\hline
Pedalcycle & 17\\
\hline
School Bus & 17\\
\hline
Truck/Trailer & 17\\
\hline
Commercial Bus & 14\\
\hline
Tractor/Semi-Trailer & 14\\
\hline
Taxicab & 10\\
\hline
Single Unit Truck (3 Or More Axles) & 9\\
\hline
Other Bus & 8\\
\hline
Heavy Truck & 5\\
\hline
Pedestrian & 5\\
\hline
Motor Home/Recreational Vehicle & 2\\
\hline
Activity Bus & 1\\
\hline
EMS Vehicle, Ambulance, Rescue Squad & 1\\
\hline
Truck/Tractor & 1\\
\hline
\end{tabular}
\end{table}

Les principales types voitures impliqués dans des accidents de vélos
sont des voitures familiales, voitures des port, Pickup et Van. Ceci
pourrait s'expliquer par le fait que les cycliste se promenant surtout
en agglomération ils rencontre surtout ce genre de voiture.

\hypertarget{etude-sur-les-donnuxe9es-temporelles}{%
\section{Etude sur les données
temporelles}\label{etude-sur-les-donnuxe9es-temporelles}}

\begin{verbatim}
## `stat_bin()` using `bins = 30`. Pick better value with `binwidth`.
\end{verbatim}

\includegraphics{Analyse-statistique_files/figure-latex/unnamed-chunk-25-1.pdf}

Le premier grahique nous montre que globalement au fil des années le
nombre d'accident impliquant un cycliste et un conducteur est resté plus
ou moins le même au alentours de 1000 accident sauf au cours de l'année
2009 où nous avons eu une nette diminution d'environ 200 accidents.

Nous avons des maintenant des informations concernant l'heure, le
mois(extraite des la variable date) et l'année des differents accidents.

\hypertarget{evolution-du-nombre-daccident-par-mois-en-fonction-des-annuxe9es}{%
\subsection{Evolution du Nombre d'accident par mois en fonction des
années}\label{evolution-du-nombre-daccident-par-mois-en-fonction-des-annuxe9es}}

\includegraphics{Analyse-statistique_files/figure-latex/unnamed-chunk-27-1.pdf}

Les courbes representative de l'evolution des accidents en fonction des
mois, par année , se chevauchant entre elles et etant données qu'aucune
ne se démarque des autres, on peut supposer que le nombre d'accident est
rester tendanciellemnt le même au cours de ces 5 années. On remarque
aussi que pour pratiquement toutes les années le nombre d'accident est
plus élevé entre le 6eme et le 9eme mois c'est a dire de juin à
septembre.

\hypertarget{carte}{%
\section{Carte}\label{carte}}

\includegraphics{Analyse-statistique_files/figure-latex/unnamed-chunk-28-1.pdf}
\#\# Zoom

Faisons un zoom sur les 4 comtés où le nombre d'accident est le plus
important. Elles se trouve dans la meme zone géographique, l'est du
Pays.

\includegraphics{Analyse-statistique_files/figure-latex/unnamed-chunk-29-1.pdf}
```

\end{document}
